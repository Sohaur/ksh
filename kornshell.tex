\documentclass{article}
\usepackage{graphicx}
\usepackage[hungarian]{babel}
\usepackage{xcolor}
\usepackage{listings}
\lstset{basicstyle=\ttfamily,
	showstringspaces=false,
	commentstyle=\color{red},
	keywordstyle=\color{blue}
}

\begin{document}

\title{Introduction to KornShell}
\author{Soha Péter}
\date{\today}

\maketitle

\begin{abstract}
Csak saját felelősségre! KSH szakértők \textbf{NE} olvassák!
\end{abstract}

\section{Részletes dolgok}
\textbf{\$0}: az éppen végrehajtott script relativ útvonalát tartalmazza az aktuális könyvtárhoz képest.\newline

\textbf{\$\{.sh.file\}}: KornShell specifikus cucc,  majdnem ugyanazt tudja, mint a \textbf{\$0}, csak ez abszolút path-t ad vissza\newline

\textit{variable}=\textbf{\$(script)}: végrehajtja az argumentumában lévő parancsokat és a printoutját behelyettesiti a \textit{variable} változóba.\newline

\textbf{pwd}: \textbf{P}rint \textbf{W}orking \textbf{D}irectory. Az aktuális könyvtár \textit{abszolút} elérési útját adja vissza (alapértelmezetten printeli).\newline

\textbf{source vagy .} \textit{fájlnév}: "Bemásolja" futásidőben a mögötte lévő fájl tartalmát az aktuális scipt "elejére". Gyakorlatilag, mint a \textit{java import} vagy \textit{c++ include}.\newline 

\textbf{Ez az sor}: 
"\texttt{[[ \textit{feltetel} ]]\&\& utasitas1 $||$ utasitas2}" ezt jelenti: 
\begin{lstlisting}[language=bash]
if [feltetel]; then
utasitas1 
else
utasitas2
fi
\end{lstlisting}
\textit{fontos!} Az utasitas1 mindenképp igazzal kell visszatérjen, mert ellenkező esetben a \texttt{logikai vagy} miatt az utasitas2 is végrehajtódna ( mivel a \texttt{logikai és} mindkét operadusának igaznak kell lennie, vagy az eredmény hamis).\newline

\textbf{read}: beletesz egy \textit{printout}-ot egy változóba. Szintaxisa: \texttt{dirname \$0 $|$ read valtozonev}. Ez ekvivalens azzal, hogy: \texttt{valtozo=\$(dirname \$)}\newline

\texttt{stream redirect}: \textit{echo} "valami" [$>\&1$ / $>\&2$]: "valami kiirása a standard outputra / error streamre" 

\textbf{Stringekről:} Stringeket úgy lehet definiálni, hogy:
\begin{itemize}
	\item "\$xxx" (Ebben az esetben minden, ami kiértékelhető, ki lesz értékelve)
	\item '\$xxx' (Ebben az esetben minden \textbf{SZÓ SZERINT} kerül a stringbe, semmi kiértékelés)
	\item $`alma`$ Ez azzal ekvivalens, hogy \textbf{\$()}
\end{itemize}


\newpage
\section{Okosságok}
\begin{itemize}
	\item korn shebang: \textbf{\#!/bin/ksh} (utána kötelező üres sor!)
	\item A változók nem tartalmaznak perjelet!
	\item printout berakása változóba: \texttt{w=\$(cd \$(dirname \$0));pwd}
	\item függvény szintaxis 1: \textbf{function} \texttt{name} \{\}
	\item függvény szintaxis 2: \texttt{name}()\{\} (nem használjuk!!!)
	\item if szintaxis 1: \textbf{if}[ \textit{feltétel} ]; then \textit{command} \textbf{fi}
	\item if szintaxis 2: [[ "\$valami" ]] (impliciten -n flaget ért alatta a ksh)
	\item -z/-n a feltételben: nulla/nemnulla a hossz
	\item a [[ \$VALTOZO ]]: \$Valtozo hossza nem nulla (ugyanaz, mint: [[-n \$Valtozo]])
	\item \$IFS/\$OFS: input és output field separator. Default whitespace, átállitható, de nem ajánlott
	\item a \textbf{read}-del több változóba is be lehet töltetni cuccot!
	\item \&\&: logikai \texttt{és} művelet
	\item $||$: logikai \texttt{vagy} művelet
	\item \texttt{file descriptor 1}: standard out
	\item \texttt{file descriptor 2}: standard error
	\item a \textbf{return} utasitás csak numerikus értéket tud visszaadni 0-256-ig
	\item \textbf{std error átirányitása az std outra}: \textit{$2>\&1$}
	
\end{itemize}




\end{document}